% !TEX root = main.tex

\documentclass{article}
\usepackage[T1]{fontenc}
\usepackage[utf8]{inputenc}
\usepackage{amsmath}
\usepackage{amssymb}
\usepackage{hyperref}
\usepackage{parskip} %skip the indent of a new paragraph.
\usepackage{float}
\usepackage{graphicx}
\usepackage{listings}
\usepackage[per-mode=symbol]{siunitx}
\usepackage{epstopdf}
\usepackage[super]{nth}
\lstset{language=Matlab, frame=single, breaklines=true,numbers=left, keywordstyle=\color{blue},rulecolor=\color{black},commentstyle=\color{gray}}

\usepackage{cleveref}
\usepackage{todonotes}

%% Brukes for tabeller (av likninger)
\usepackage{tabularx}
\def\tabularxcolumn#1{m{#1}}

\newcommand{\mbf}[1]{\mathbf[#1]}
\newcommand{\partialderiv}[2]{\frac{\partial{#1}}{\partial{#2}}} % prints partial derivative as a fraction

\begin{document}
    % !TEX root = main.tex

\begin{titlepage}
    %\maketitle
    %\rule{\linewidth}{0.5mm}
    \begin{center}
    	\large
    	Real Time Programming TTK4145
    \end{center}
    \vspace{\fill}
    \rule{\linewidth}{0.5mm}
    \begin{center}
    	\huge
    	Design
    \end{center}
	\rule{\linewidth}{0.5mm}
	\vspace{\fill}

	%\begin{center}
    %	\huge
    %	Bern Johan Damslora -- 759477 \\ Didrik Rokhaug -- 759528
    %\end{center}
    \large
    \centering
    \begin{table}[H]
    	\centering
    	\large
    	%\begin{tabular}{rl}
    		\textbf{Didrik Rokhaug}\\
            \textbf{Bern Johan Damslora}
    	%\end{tabular}
    \end{table}
    \vspace{\fill}
    \begin{center}
    	\large
    	\today
    \end{center}
	\vspace{\fill}
    \begin{figure}[H]
    \centering
    \includegraphics[width=0.5\textwidth]{logontnu_eng}
    \end{figure}
    \thispagestyle{empty}
\end{titlepage}

\end{document}
